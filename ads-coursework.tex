\documentclass[11pt,a4paper]{article}

\title{Advanced Database Systems Coursework}
\author{Ross Fenning}

\begin{document}

\maketitle

\section{Introduction}

The aim of this report is to explore and compare the different properties of
three different kinds of databases beyond the traditional relational model. In it,
the same dataset will be modelled in each of an \emph{Object-Relational} database,
an \emph{XML} database and a \emph{Document} database.

An optimal structure or schema will be designed for each database type, then
each design will be implemented in each of \emph{PostgreSQL}, \emph{eXist}
and \emph{MongoDB} along with a discussion around how each database would be
queried for different use cases.

Finally, there will be a comparitive discussion around the advantages and
disadvantages between the different database types, including which the use
cases to which each kind of database are most appropriate.

\section{The Data}

The dataset that will be used is a small set of 7,788 TV and radio programmes
that have been available to listen to or watch on some form of catch-up
service (e.g. the old radio ``Listen Again'' service or BBC iPlayer) between
2001 and 2012.

The raw data begin in a comma-separated format, with each row representing
a single programme. Each programme row comprises:

\begin{itemize}
    \item an identifier known as a \emph{pid} (programme ID);
    \item a start and end time (in both Unix timestamp and ISO 8601 formats);
    \item a title;
    \item a media type (i.e. audio or video);
    \item the master brand with which the programme is associated (somewhat
      correlated with the channel that broadcasted the programme, e.g.
      BBC Radio Four or BBC Two);
    \item the service that did broadcase the programme originally;
    \item a pid identifier for the programme's brand (e.g. for an episode of
      Doctor Who this would be the identifier for the Doctor Who concept
      in general);
    \item whether or not this is a programme or a just a clip that was
      available to watch/listen to;
    \item any categories associated with the programme; and
    \item any tags set against the programme.
\end{itemize}

\section{Object-Relational}
\subsection{Design}
\subsection{Implementation}
\subsection{Querying}

\section{XML Database}
\subsection{Design}
\subsection{Implementation}
\subsection{Querying}

\section{Document Database}
\subsection{Design}
\subsection{Implementation}
\subsection{Querying}

\section{Comparison}

\end{document}
